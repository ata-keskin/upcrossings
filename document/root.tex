\documentclass[11pt,a4paper]{article}
\usepackage[T1]{fontenc}
\usepackage{isabelle,isabellesym}

% further packages required for unusual symbols (see also
% isabellesym.sty), use only when needed

\usepackage{amssymb}
\usepackage{amsmath}
  %for \<leadsto>, \<box>, \<diamond>, \<sqsupset>, \<mho>, \<Join>,
  %\<lhd>, \<lesssim>, \<greatersim>, \<lessapprox>, \<greaterapprox>,
  %\<triangleq>, \<yen>, \<lozenge>

%\usepackage{eurosym}
  %for \<euro>

%\usepackage[only,bigsqcap,bigparallel,fatsemi,interleave,sslash]{stmaryrd}
  %for \<Sqinter>, \<Parallel>, \<Zsemi>, \<Parallel>, \<sslash>

%\usepackage{eufrak}
  %for \<AA> ... \<ZZ>, \<aa> ... \<zz> (also included in amssymb)

%\usepackage{textcomp}
  %for \<onequarter>, \<onehalf>, \<threequarters>, \<degree>, \<cent>,
  %\<currency>

% this should be the last package used
\usepackage{pdfsetup}

% urls in roman style, theory text in math-similar italics
\urlstyle{rm}
\isabellestyle{it}

% for uniform font size
%\renewcommand{\isastyle}{\isastyleminor}


\begin{document}

\title{Upcrossings}
\author{Ata Keskin}
\maketitle

\begin{abstract}
In the scope of this entry, we formalize Doob's Upcrossing Inequality and subsequently prove Doob's first martingale convergence theorem. 

Doob's upcrossing inequality is a fundamental result in the study of martingales. It provides a bound on the expected number of times a submartingale crosses a certain threshold within a given interval.

The theorem states that, if $(M_n)_{n \ge 0}$ is a submartingale with $\sup_n \mathbb{E}[M^{+}_n] < \infty$, then, the limit process $M_\infty := \lim_n M_n$ exists almost surely and is integrable. Furthemore the limit process is measurable with respect to $F_\infty = (\bigcup_{n \ge 0}. F_n)$. Equivalent statements for martingales and supermartingales are also provided as corollaries.

The proofs provided are based mostly on the formalization done in the Lean mathematical library \cite{ying2022formalization}.
\end{abstract}

\tableofcontents

% sane default for proof documents
\parindent 0pt\parskip 0.5ex

% generated text of all theories
\input{session}

% optional bibliography
\bibliographystyle{abbrv}
\bibliography{root}

\end{document}

%%% Local Variables:
%%% mode: latex
%%% TeX-master: t
%%% End:
